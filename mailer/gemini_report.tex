\documentclass{article}%
\usepackage[T1]{fontenc}%
\usepackage[utf8]{inputenc}%
\usepackage{lmodern}%
\usepackage{textcomp}%
\usepackage{lastpage}%
%
\title{Gemini Vocabulary Report}%
\date{2025{-}06{-}09}%
%
\begin{document}%
\normalsize%
\maketitle%
\section{Time for reflection}%
\label{sec:Timeforreflection}%
```latex
\documentclass{article}
\usepackage{amsmath}

\begin{document}

\section*{Vocabulary Words}

\textbf{reflection}: serious thought or consideration.

\textbf{unity}: the state of being united or joined as a whole.

\textbf{oppressive}: unjustly inflicting hardship and constraint, especially on a minority or other subordinate group.

\textbf{enduring}: lasting; continuing.

\textbf{genocide}: the deliberate killing of a large number of people from a particular nation or ethnic group with the aim of destroying that nation or group.

\textbf{belligerence}: aggressive or warlike behavior.

\textbf{cleric}: a priest or religious leader.

\textbf{divine deliverance}: rescue or liberation by God.

\textbf{substance}: importance, meaningfulness, or value.

\textbf{butchery}: the savage killing of a large number of people.

\textbf{aggression}: hostile or violent behavior or attitudes toward another; readiness to attack or confront.

\textbf{flagrant}: conspicuously or obviously offensive.

\textbf{norms}: something that is usual, typical, or standard.

\textbf{averted}: turn away (one's eyes or thoughts).

\textbf{spectre}: something widely feared as a possible unpleasant or dangerous occurrence.

\textbf{stalk}: pursue or approach stealthily.

\textbf{militancy}: the use of confrontational or violent methods in support of a political or social cause.

\textbf{egalitarian}: believing in or based on the principle that all people are equal and deserve equal rights and opportunities.

\textbf{conscientious}: (of a person) wishing to do what is right, especially to do one's work or duty well and thoroughly.

\textbf{blockade}: an act or means of sealing off a place to prevent goods or people from entering or leaving.

\textbf{integrity}: the quality of being honest and having strong moral principles.

\textbf{monstrous}: shockingly brutal or cruel.

\textbf{compassion}: sympathetic pity and concern for the sufferings or misfortunes of others.

\textbf{lofty}: of imposing height.

\textbf{brotherhood}: the feeling of kinship with and responsibility for other people.

\textbf{decency}: behavior that conforms to accepted standards of morality or respectability.

\section*{Phrases and Idioms}

\textbf{in the shadow of}: overshadowed by something negative.

\textbf{making a mockery of}: to treat someone or something with disrespect and make them seem ridiculous.

\textbf{tut-tutting}: expressing disapproval or annoyance.

\textbf{big picture}: the overall perspective or context.

\textbf{stalk the land}: to move through a place like a hunter.

\textbf{cut ties with}: end a relationship or connection.

\end{document}
```

%
\section{Pushed into poverty}%
\label{sec:Pushedintopoverty}%
```latex
\documentclass{article}
\usepackage{amsmath}

\begin{document}

\section*{Vocabulary Words}

\textbf{Proportion}: A part, share, or number considered in comparative relation to a whole. \\
\textbf{Devastating}: Highly destructive or damaging. \\
\textbf{Inflation}: A general increase in prices and fall in the purchasing value of money. \\
\textbf{Trends}: A general direction in which something is developing or changing. \\
\textbf{Assessment}: The evaluation or estimation of the nature, quality, or ability of someone or something. \\
\textbf{Dimensions}: A measurable extent of something. \\
\textbf{Spatial}: Relating to or occupying space. \\
\textbf{Disparities}: A great difference. \\
\textbf{Fiscal}: Relating to government revenue, especially taxes. \\
\textbf{Equity}: The quality of being fair and impartial. \\
\textbf{Threshold}: A level or point at which something would start or cease to happen or come into effect. \\
\textbf{Stagnant}: (of water or air) not flowing or moving. \\
\textbf{Socioeconomic}: Relating to or concerned with the interaction of social and economic factors. \\
\textbf{Interventions}: The action or process of intervening. \\

\section*{Phrases and Idioms}

\textbf{In the wake of}: Following after or as a result of something. \\
\textbf{Poverty line}: The estimated minimum level of income needed to secure the necessities of life. \\
\textbf{Due in}: Expected to arrive or be completed at a specified time. \\
\textbf{Political will}: The intent and determination to take action to deal with a political problem or situation. \\
\end{document}
```

%
\section{Detention law}%
\label{sec:Detentionlaw}%
```latex
\section*{Vocabulary Words}
\textbf{Presumed}: to assume something is true until proven otherwise.

\textbf{Enshrined}: to preserve or protect something valued.

\textbf{Preventive detention}: imprisonment of a person with the purpose of preventing future crimes or other actions.

\textbf{Ostensibly}: apparently or purportedly, but perhaps not actually.

\textbf{Sweeping}: wide in range or effect.

\textbf{Vague}: not clearly expressed or defined.

\textbf{Judicial}: relating to courts or judges.

\textbf{Formalised}: to give something legal or official status.

\textbf{Intractable}: hard to control or deal with.

\textbf{Grievances}: a real or imagined wrong or other cause for complaint or protest, especially unfair treatment.

\textbf{Palatable}: pleasant to taste; also, acceptable or agreeable.

\section*{Phrases and Idioms}
\textbf{Run out of ideas}: to have no more ideas.

\textbf{Fuel to the fire}: to make a bad situation worse.

\textbf{Legal cover}: protection from legal consequences.
```

%
\end{document}